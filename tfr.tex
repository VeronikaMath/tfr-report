\documentclass{article}  % this line must be in main document!! otherwise compiler gets confused

\usepackage[utf8]{inputenc}
% \usepackage{natbib}
\usepackage{physics}
\usepackage{biblatex}
\usepackage[hidelinks]{hyperref}

\usepackage[table,xcdraw]{xcolor}
\usepackage{svg}
\usepackage[T1]{fontenc}
\usepackage{amsmath}
\usepackage{amssymb}
\usepackage{dsfont}
\usepackage{lmodern}
\usepackage{units}
\usepackage{icomma}
\usepackage{graphicx}
\usepackage{eurosym}
\usepackage{verbatim}
\usepackage[margin=3.4cm]{geometry}
\usepackage{float}
\usepackage{listings}
\usepackage{mathtools}
\usepackage{caption}
\usepackage{color} %red, green, blue, yellow, cyan, magenta, black, white
\definecolor{mygreen}{RGB}{28,172,0} % color values Red, Green, Blue
\definecolor{mylilas}{RGB}{170,55,241}
\setcounter{MaxMatrixCols}{30}
\definecolor{dkgreen}{rgb}{0,0.6,0}
\definecolor{gray}{rgb}{0.5,0.5,0.5}
\definecolor{mauve}{rgb}{0.58,0,0.82}
\usepackage{textcomp}
\usepackage[titletoc,title]{appendix}
\DeclareMathOperator*{\argmax}{arg\,max}
\DeclareMathOperator*{\argmin}{arg\,min}
\allowdisplaybreaks
%
\lstset{language=Matlab,%
    %basicstyle=\color{red},
    breaklines=true,%
    morekeywords={matlab2tikz},
    keywordstyle=\color{black},%
    morekeywords=[2]{1}, keywordstyle=[2]{\color{black}},
    identifierstyle=\color{black},%
    stringstyle=\color{mylilas},
    commentstyle=\color{mygreen},%
    showstringspaces=false,%without this there will be a symbol in the places where there is a space
    numbers=left,%
    numberstyle={\tiny \color{black}},% size of the numbers
    numbersep=9pt, % this defines how far the numbers are from the text
    emph=[1]{for,end,break},emphstyle=[1]\color{black}, %some words to emphasise
    %emph=[2]{word1,word2}, emphstyle=[2]{style},    
}
\newcommand{\innerprod}[2]{\langle #1, #2 \rangle}
\addbibresource{bibliography.bib}

\title{TFR Notes}
\author{Veronika Chronholm}
\date{\today}

\begin{document}

\maketitle

\section{Introduction}

\section{Theory}

% - PDE Constrained optimisation problem 

% - Forward + Backward PDEs

% - Initial/terminal conditions, boundary conditions, and source terms (figure out equivalent formulations?)

% \vspace{5mm}
% - Forward + Backward SDEs

% - Connections between FBSDEs and the PDEs (eg distribution of process given by forward SDE is described by sol to Fokker-Planck/Kolmogorov forward eqn)

% - Feynman-Kac formulae (of various kinds)

% - Solving Fokker-Planck using MC + Feynman-Kac (as in reading course)

% - Simplest scheme for Forward-Backward SDEs

% - New (\cite{fang2023strong}) SSP scheme for FBSDEs

% - Specifically what schemes and equations look like for the reading course model

% - Similarities/Differences between the "first" (linear?) kind of Feynman-Kac formula, and the FBSDE *nonlinear* Feynman-Kac formulae?

% - CONSISTENT NOTATION --- there is something weird with functions $f$ and $g$ at the moment, for example\dots

\subsection{Forward and Backward PDEs}

Treatment planning problem (PDE constrained optimisation):
%
\begin{align} 
    \min \frac{1}{2} {\lVert u - d_T \rVert}^2 + \frac{\alpha}{2} {\lVert g \rVert}^2
\end{align}
%
subject to a constraint
%
\begin{align} 
    &\partial_t u + b \cdot \nabla u - \mu \Delta u = g\\
    &{u \rvert}_{\partial \Omega} = 0
\end{align}
%
Using the method of Lagrange multipliers, we can solve the above optimisation problem by solving simultaneously the two PDEs
%
\begin{align}
    - \partial_t z - \mu \Delta z - b \cdot \nabla z &= u - d_T\\
    \partial_t u - \mu \Delta u + b \cdot \nabla u &= \frac{1}{\alpha} z 
\end{align}
%

\subsection{Deriving coupled PDEs from the treatment planning problem}

% - PDE constrained optimisation problem 

% - Introduce our special case model

% - But, what norm do we use? Introduce Bochner spaces (and this particular one also has an inner product, which is important)

% - Write down the Lagrangian

% - Functional derivatives = zero yields three equations, which become the two coupled PDEs

% - Comment on how the problem would change if we changed the norm we're minimising in?

We formulate the treatment planning problem as a constrained optimisation problem, where the constraint is a PDE describing the evolution of radiation in the domain. If we let $u$ denote the radiation intensity in the domain, and $d_T$ the target intensity, we seek the function $u$ that gives
%
\begin{align} 
    \label{eq:to-minimise}
    \min \frac{1}{2} {\lVert u - d_T \rVert}^2 + \frac{\alpha}{2} {\lVert g \rVert}^2
\end{align}
%
for some constant $\alpha$, and subject to the constraint
%
\begin{align}
    \label{eq:pde-constraint}
    &\partial_t u + \div{(\underline{b} u + A \nabla u)} = g\\
    &{u \rvert}_{\partial \Omega} = 0.
\end{align}
%
The constraint describes the physical evolution of radiation in the domain. Here, we are specifically interested in describing radiation in terms of its evolution as a function of time $t$, spatial position $x$, and the velocity $v$ of the radiation particles. If we only consider one spatial dimension, our problem will have a total of two dimensions in addition to time, i.e. the domain $\Omega=\mathbb{R}\cross\mathbb{R}$ is two-dimensional, with one dimension being position and the other velocity. Thus, $u(t,x,v): \mathbb{R}^{+} \cross \Omega \rightarrow \mathbb{R}$.

Specifically, we will consider a model where
%
\begin{equation}
\underline{b} = 
\begin{pmatrix}
    &v\\
    &-a(v)
\end{pmatrix},
A = 
\begin{pmatrix}
    &0 &0\\
    &0 &-\frac{\sigma^2}{2}
\end{pmatrix},
\end{equation}
%
meaning that we only have a second derivative with respect to $v$, and not with respect to $x$.

So far, we have formulated this optimisation problem without specifying which norm is used in \autoref{eq:to-minimise}. An intuitive choice may be the $L^2$ norm over the domain $\Omega$, but if we choose this norm we haven't taken into account the time dependence of $u$. With this in mind, we introduce the more general concept of a Bochner space $L^p(T;X)$ and the corresponding norm
%
\begin{align} 
    {\lVert (\cdot) \rVert}^p_{L^p(T,X)} := \int_{T} {\lVert (\cdot)\rVert}^p_X dt, 
\end{align}
%
where $X$ is a Banach space with corresponding norm ${\lVert \cdot\rVert}_X$. Specifically, we require the case where $T$ is a time interval and $X=L^2(\Omega)$. A suitable choice for the norm to minimise \autoref{eq:to-minimise} in is then the following;
%
\begin{align} 
    \lVert f \rVert_{L^2([0,T];L^2(\Omega))} &:= \int_{0}^{T}  {\lVert f \rVert}^2_{L^2(\Omega)} dt\\
    &= \int_{0}^{T} \int_{\Omega} f^2 d\Omega dt.\\
\end{align}
%
We note that the space $L^2([0,T];L^2(\Omega))$ additionally has an inner product, namely
%
\begin{align} 
    \innerprod{f}{g}_{L^2(T;L^2(\Omega))} &:= \int_{0}^{T} \innerprod{f}{g}_{L^2(\Omega)} dt \\
    & = \int_{0}^{T} \int_{\Omega} fg d\Omega dt.
\end{align}
%
This inner product will be used in what follows.

We now proceed to treat the minimisation problem, using the method of Lagrange multipliers. First, we formulate the Lagrangian function for the optimisation problem specified by \autoref{eq:to-minimise} and \autoref{eq:pde-constraint}. Then, we take generalised derivatives of the Lagrangian function and set these to zero. Finally, from the resulting equations we obtain a set of two coupled PDEs.

The Lagrangian function is given by
%
\begin{align} 
    \mathcal{L}(u,g,z) := \frac{1}{2} {\lVert u - d_T \rVert}^2 + \frac{\alpha}{2} {\lVert g \rVert}^2 - \innerprod{u}{\partial_t z} - \innerprod{u}{\underline{b} \cdot \nabla z} + \innerprod{\nabla u}{A \nabla z} - \innerprod{g}{z},
\end{align}
%
where we use $\lVert \cdot \rVert$ and $\innerprod{\cdot}{\cdot}$ to denote the norm and inner product in $L^2([0,T];L^2(\Omega))$, here and in what follows. The function $u$ is the function that we seek in the minimisation problem, $g$ is the source term from the constraining PDE, and $z$ is another function known as the Lagrange multiplier, which is introduced to capture the constraint.

By taking generalised (functional) derivatives of $\mathcal{L}$ with respect to $u$,$z$, and $g$ we obtain the following;
%
\begin{align} 
    D_u \mathcal{L}[\phi] &= \int_{0}^{T} \int_{\Omega} \Big((u-d_T)\phi - \partial_t z \phi - \phi \underline{b}\cdot\nabla z + A \nabla z \cdot \nabla \phi \Big) d\Omega dt\\
    &= \int_{0}^{T} \int_{\Omega} \Big((u-d_T)\phi - \partial_t z \phi - \phi \underline{b}\cdot\nabla z + \div{(A \nabla z)} \phi \Big) d\Omega dt,
\end{align}
%
\begin{align} 
    D_z\mathcal{L}[\phi] = \int_{0}^{T} \int_{\Omega} \Big( \phi \partial_t u + \div{(\underline{b}u)} \phi - \div{(A \nabla u)} \phi - g \phi \Big) d\Omega dt,
\end{align}
%
\begin{align} 
    D_g\mathcal{L}[\phi] = \int_{0}^{T} \int_{\Omega} \Big( \alpha g \phi - z \phi \Big) d\Omega dt,
\end{align}
%
where $\phi = \phi(t,x,v)$ is a suitable test function.

By requiring $D_u \mathcal{L}[\phi]=D_z \mathcal{L}[\phi]=D_g \mathcal{L}[\phi]=0$ we get a PDE for $z$;
%
\begin{align} 
    - \partial_t z - \underline{b} \cdot \nabla z - \div{(A \nabla z)} + (u - d_T) = 0,
\end{align}
% 
a PDE for $u$;
%
\begin{align} 
    \partial_t u + \div{(\underline{b}u)} - \div{(A \nabla u)} - g = 0,
\end{align}
%
which we note recovers the constraint \autoref{eq:pde-constraint}, and finally a relationship between $z$ and $g$;
%
\begin{align}
    \alpha g - z = 0. 
\end{align}
%
Putting these three equations together, we obtain the coupled PDEs 
%
\begin{align} 
    \begin{cases} 
        \partial_t u + \div{(\underline{b}u)} - \div{(A \nabla z)} &= \frac{1}{\alpha} z\\
        \partial_t z + \underline{b} \cdot \nabla z + \div{(A \nabla z)} &= u - d_T,
    \end{cases}
\end{align}
%
where the source term of the equation for $u$ depends on $z$, and vice versa.

\subsection{Forward SDE and linear Feynman-Kac formula}

Let $b:\mathbb{R}^+ \times \mathbb{R}^d \rightarrow \mathbb{R}^d$, $\sigma:\mathbb{R}^+ \times \mathbb{R}^d \rightarrow \mathbb{R}^{d \times d}$, and $W_t$ a $d$-dimensional Brownian motion. Consider the $d$-dimensional SDE
%
\begin{align}
    \label{eq:gen-sde}
X_t = x + \int_0^t b(s,X_s) ds + \int_0^t \sigma(s,X_s)dW_s
\end{align}
%
for the process $X_t$. 

The infinitesimal generator of $X_t$ is the differential operator 
%
\begin{align}
    \label{eq:generator-operator}
    \mathcal{L} = \frac{1}{2} \sum_{i,j=1}^d [\sigma \sigma^T]_{i,j}(t,x) \partial_{x_i,x_j}^2 + \sum_{i=1}^d b_i(t,x) \partial_{x_i}.
\end{align}
%
This operator is defined by
%
\begin{align} 
    \mathcal{L}f(x) = \lim_{t\rightarrow 0} \frac{\mathbb{E}_x[f(X_t)]-f(x)}{t}.
\end{align}
%

Below, we describe the connection between a PDE involving this differential operator, and the solution to the SDE \autoref{eq:gen-sde}, yielding a probabilistic representation of the solution to the PDE. We consider the terminal value problem
%
\begin{align}
    \label{eq:backward-pde}
    \begin{cases}
    &\partial_t u(t,x) + \mathcal{L}u(t,x) - k(t,x)u(t,x) + g(t,x) = 0, \quad t<T,x \in \mathbb{R}^d, \\
    &u(T,x) = f(x).
    \end{cases}
\end{align}
%
We note that the probabilistic convention is to formulate this problem as a terminal value problem, as we have done here. However, the PDE convention would be to instead formulate it as an initial value problem. These two formulations are equivalent under a time-reversal $t \mapsto T-t$. 

Under some conditions (see for example \cite{gobet2016monte} for the specifics) on the functions $f, g, k$ as well as on $b$ and $\sigma$, and provided that the solution $u$ to \autoref{eq:backward-pde} exists and is in $\mathcal{C}^{1,2}$, and satisfies some further conditions of continuity and boundedness, $u(t,x)$ is given by the Feynman-Kac formula
%
\begin{align}
    \label{eq:linear-fc} 
    u(t,x) = \mathbb{E}\bigg[ f\big( X_T^{t,x} \big) e^{-\int_t^T k(r,X_r^{t,x})dr} + \int_t^T g(s,X_s^{t,x}) e^{-\int_t^s k(r,X_r^{t,x})dr}ds\bigg].
\end{align} 
%
Here $X_T^{t,x}$ denotes the stochastic process $X$ at time $T$, started at $x$ at time $t$.

We note that in particular, for $k=0,g=0$, the solution to the Kolmogorov backward equation (with terminal condition),
%
\begin{align}
    \label{eq:kolmogorov-backward}
    \begin{cases}
    &\partial_t u(t,x) + \mathcal{L}u(t,x) = 0, \quad t<T,x \in \mathbb{R}^d, \\
    &u(T,x) = f(x),
    \end{cases}
\end{align}
%
is given by 
%
\begin{align}
    u(t,x) = \mathbb{E}\big[ f(X_T^{t,x}) \big]. 
\end{align} 
% 
Another PDE related to the SDE \autoref{eq:gen-sde} is the Kolmogorov forward equation
%
\begin{align} % sign?
    \label{eq:kolmogorov-forward}
    \partial_s p(s,y) - \mathcal{L}^{*}p(s,y) = 0,
\end{align}
%
where the differential operator $\mathcal{L}^{*}$ is the adjoint of $\mathcal{L}$, given by
%
\begin{align}
    \label{eq:adjoint-operator}
     \mathcal{L}^{*} = \frac{1}{2} \sum_{i,j=1}^{d} \partial_{y_i,y_j} [\sigma \sigma^T]_{i,j}(s,y) - \sum_{i=1}^{d} \partial_{y_i}b_i(s,y).
\end{align} 
%
% is the IC at time zero or time t? Probably zero...
The Kolmogorov forward equation (with initial condition $p(0,y)=\delta(y-x)$) describes the probability distribution of the stochastic process $X_t$ that solves the SDE in \autoref{eq:gen-sde}. We can obtain a Feynman-Kac type formula for the density function $p(s,y)$ (assuming that $X_t$ admits a density), by noting that
%
\begin{align} 
    \mathbb{E}[f(X_T^{t,x})] = \int p_{x,t}(T,z)f(z)dz 
\end{align}
%
Hence,
%
\begin{align}
    p(T,y) = \mathbb{E}[\delta({X_T^{t,x} - y})] = \int p_{t,x}(T,z)\delta({z-y})dz, 
\end{align} 
%
or more generally
%
\begin{align}
    p(s,y) = \mathbb{E}[\delta({X_s^{t,x} - y})] = \int p_{t,x}(s,z)\delta({z-y})dz.
\end{align} 
%
There are some subtleties that have not been covered here --- e.g.\ what is required for $X_t$ to admit a density function, and the regularity required for the expectation of an indicator function to make sense. We also note that when writing $p(s,y)$, the probability density function of $X$ being at a point $y$ at time $s$, we are implicitly referring to the probability density conditioned on the initial distribution, i.e. in this case conditioned on $X$ starting at position $x$ at time $t$, for some $t<s$.

\subsection{Forward Backward SDE and nonlinear Feynman-Kac formula}
Now, we introduce the forward-backward SDE
%
\begin{align} 
    \label{eq:fbsde}
    \begin{cases}
    X_t &= x + \int_0^t b(s,X_s) ds + \int_0^t \sigma(s,X_s)dW_s\\
    & \\
    Y_t &= f(X_T) + \int_t^T g\big(s,X_s,Y_s,Z_s[\sigma(s,X_s)]^{-1}\big)ds - \int_t^T Z_s dW_s,
    \end{cases}
\end{align}
%
where the first equation is the forward SDE --- identical to \autoref{eq:gen-sde} --- and the second equation is the backward SDE.\@ We note that $X_t$ depends on the values of $X$ prior to time $t$, whilst $Y_t$ depends on the values of $X,Y,Z$ after time $t$ (and up to time $T$). 

It can be shown that the backward SDE above is related to the PDE (terminal value problem)
%
\begin{align}
    \label{eq:nonlinear-pde}
    &\partial_t u(t,x) + \mathcal{L}u(t,x) + g(t,x,u(t,x),\nabla u(t,x)) = 0 \\
    &u(T,x) = f(x), 
\end{align} 
%
where, as before, the operator $\mathcal{L}$ is the infinitesimal generator of the forward SDE. Specifically, if the solution to the terminal value problem exists, then the processes $Y_t$, $Z_t$ given by
%
\begin{align} 
    Y_t &= u(t,X_t)\\
    Z_t &= \sigma(t,X_t) \nabla u(t,X_t)
\end{align}
%
satisfy the FBSDE of \autoref{eq:fbsde}. We also note that this statement can be extended to a system of $k$ PDEs, and a vector valued stochastic process $Y_t$. 

Similarly to in \autoref{eq:linear-fc}, we can express the solution $u(t,x)$ to the terminal value problem \autoref{eq:nonlinear-pde} in terms of a (now nonlinear) Feynman-Kac formula as
%
\begin{align} 
    \label{eq:nonlinear-fc}
    u(t,x) = \mathbb{E}\bigg[ f(X_T^{t,x}) + \int_t^T g\big(s,X_s^{t,x},u(s,X_s^{t,x}),\nabla u(s,X_s^{t,x})\big)ds \bigg].
\end{align} 
%
Alternatively, we can write $Y_t$ as
%
\begin{align} 
    \label{eq:nonlinear-fc-for-y}
    Y_t = u(t,X_t) = \mathbb{E}\bigg[ f(X_T) + \int_t^T g\big(s,X_s,Y_s,Z_s\big)ds \Big\lvert X_t \bigg].
\end{align}
%

Comparing \autoref{eq:nonlinear-fc} to \autoref{eq:linear-fc}, we note that the function $g$ now in general depends on not only $t$ and $X_t$, but can also depend on $u$ and $\nabla u$. The discount (or attenuation) factor $k(t,x)$ in \autoref{eq:backward-pde} has been absorbed into the more general source term $g(t,x,u(t,x),\nabla u(t,x))$ in \autoref{eq:nonlinear-pde}. We also note that in \autoref{eq:linear-fc}, only the left hand side depends on $u$, whilst in \autoref{eq:nonlinear-fc} the right hand side also depends on $u$. Hence, to use the latter for numerical simulation of $u(t,x)$, more careful consideration is required.

We have now seen that the solution to the terminal value problem (backward PDE) of \autoref{eq:nonlinear-pde} is associated with the backward SDE in \autoref{eq:fbsde}. This is a generalisation of the connection discussed in the previous subsection, and we note that by letting $g$ depend on only $t$ and $x$, we can formulate a less general backward SDE and recover a version of the linear Feymnan-Kac formula of \autoref{eq:linear-fc}, glossing over some subtleties related to the discount factor $k(t,x)$.

As before, the PDE directly associated with the forward SDE in \autoref{eq:fbsde} is the Kolmogorov forward equation of \autoref{eq:kolmogorov-forward} with initial condition $p(0,y)=\delta(y-x)$. It is not clear whether a Feynman-Kac formula can be obtained for a more general PDE featuring the operator $\mathcal{L}^*$ defined in \autoref{eq:adjoint-operator} rather than the operator $\mathcal{L}$ of \autoref{eq:generator-operator}. 

\subsection{FBSDEs and systems of PDEs}

It is also possible to write down a more general version of \autoref{eq:fbsde}, where the process $Y_t$ which solves the backward SDE is allowed to be $K$-dimensional. This process can then be used to write down a probabilistic representation of the solution to a system of $K$ coupled PDEs (each with a terminal condition) of the type written down in \autoref{eq:nonlinear-pde}. For simplicity, here we consider the case where the source term $g$ does not depend on the gradient of $u$. Specifically, we then seek a probabilistic representation of $u = (u^{[1]},...,u^{[K]})$, where 
%
\begin{align} 
    \label{eq:pde-system}
\begin{cases}
    &\partial_t u^{[k]}(t,x) + \mathcal{L}^{[k]} u^{[k]}(t,x) + g^{[k]}(t,x,u(t,x)) = 0\\
    &u^{[k]}(T,x) = f^{[k]}(x)
\end{cases}
\end{align}
%
for $k=1,2,...,K$, and $g=(g^{[1]},...,g^{[K]})$, $f=(f^{[1]},...,f^{[K]})$. We note that in general each component of $g$ depends on the whole of $u$, and not just a particular component $u^{[k]}$.

Here, as before, $\mathcal{L}^{[k]}$ denotes the generator of a diffusion process, such as that described in the forward SDE of \autoref{eq:fbsde}. If the operator $\mathcal{L}^{[k]}$ is the same for each $k$, it is enough to consider one diffusion process $X_t$. However, if $\mathcal{L}^{[k]}$ is different for each $k$, we need $K$ different diffusion processes $(X_t^{[1]},...,X_t^{[K]})$, with corresponding generators $(\mathcal{L}^{[1]},...,\mathcal{L}^{[K]})$. In general, each process $X_t^{[k]}$ is $d$-dimensional, i.e. of the same dimension as the variable $x$ in the PDE in \autoref{eq:pde-system}. 

For simplicity, here we consider the case where the $X_t^{[k]}$ are one dimensional, and have constant coefficients, but the below argument is easily generalised to the $d$-dimensional case with nonconstant coefficients. We then have
%
\begin{align} 
    X_t^{[k]} = x + \int_{0}^{t} b_k ds + \int_{0}^{t}\sigma_k dW_s^{[k]}
\end{align}
%
each with corresponding generator
%
\begin{align} 
    \mathcal{L}^{[k]} = \frac{1}{2} \sigma_k^2 \partial_x^2 + b_k \partial_x.
\end{align}
%

We now claim that the processes $Y_t = (Y_t^{[1]},...,Y_t^{[K]})$ and $Z_t=(Z_t^{[1]},...,Z_t^{[K]})$ defined by 
%
\begin{align}
    \label{eq:y-def}
    Y_t^{[k]} &= u^{[k]}(t,X_t^{[k]})\\
    Z_t^{[k]} &= \sigma_k \partial_x u^{[k]}(t,X_t^{[k]})
\end{align}
%
obey the backward SDE
%
\begin{align} 
    \label{eq:component-bsde}
    Y_t^{[k]} = f^{[k]}(X_T^{[k]}) + \int_t^T g^{[k]}\big( s, X_s^{[k]},u(s,X_s) \big)ds - \int_t^T Z_s^{[k]}dW_s^{[k]},
\end{align}
%
where $u(s,X_s) = \big( u^{[1]}(s,X_s^{[1]}),..., u^{[K]}(s,X_s^{[K]})\big)$.

We show this holds by (componentwise) applying Itô's formula
%
\begin{align} 
    v(s,X_s) = v(s_0,X_{s_0}) + \int_{s_0}^{s} [\partial_t + \mathcal{L}] v(r,X_r)dr + \int_{s_0}^{s} \partial_x v(r,X_r) \sigma(r,X_r)dW_r,
\end{align}
%
where $\mathcal{L}$ is the generator of the diffusion process given by
%
\begin{align} 
    X_s = X_{s_0} + \int_{s_0}^{s} b(r,X_r)ds + \int_{s_0}^{s}\sigma(r,X_r)dW_r,
\end{align}
%
to the process $Y_t=(Y_t^{[1]},...,Y_t^{[K]})$ defined by \autoref{eq:y-def}, with $s_0=t$ and $s=T$. By noting that $u^{[k]}(T,X_T^{[k]})=f^{[k]}(X_T^{[k]})$ the result follows.

To obtain a Feynman-Kac formula for $u(t,x)$ we take a conditional expectation of \autoref{eq:component-bsde}. What to condition on is slightly subtle. By considering that the equations for $u^{[k]}(t,x)$ for each $k$ must be equations of the same variable $x$, and are coupled through the function $g$, we can see that a sensible thing to condition on is $X_t^{[1]}=x,...,X_t^{[K]}=x$. In other words, we start each of the diffusions $X^{[k]}_s$ at the same point $x$ at time $t$.
We then get the Feynman-Kac formula
%
\begin{align} 
    u^{[k]}(t,x) = \mathbb{E}\bigg[ f^{[k]}(X_T^{[k],x,t}) + \int_{t}^{T} g^{[k]}\big( s, X_s^{[k],x,t}, u(s,X_s^{x,t}) \big)ds \bigg],
\end{align}
%
where $X_T^{[k],x,t}$ denotes the process $X_s^{[k]}$ started at $x$ at time $t$, and $X_s^{x,t}$ denotes the ($k$-dimensional) $(X_s^{[1]},...,X_s^{[K]})$ started at $(x,...,x)$ at time $t$.

Similarly to before, we can also write down a formula for $Y_t^{[k]}$ as 
%
\begin{align}
Y_t^{[k]} = \mathbb{E}\bigg[ f^{[k]}(X_T^{[k]}) + \int_{t}^{T} g^{[k]}\big( s,X_s^{[k]},Y_s\big)ds \Big\lvert X_t^{[1]},...,X_t^{[K]}\bigg],
\end{align}
%
by conditioning on the processes $(X_t^{[1]},...,X_t^{[K]})$, at time $t$, instead of on $(X_t^{[1]}=x,...,X_t^{[K]}=x)$.

\subsection{Numerical schemes for FBSDEs}

To solve the FBSDE \autoref{eq:fbsde} numerically, we need to discretise both the forward and the backward SDE in time. Discretising the forward SDE is straightforward, and can for example be done using the (forward) Euler-Maruyama scheme
%
\begin{align} 
    \begin{cases}
    &X_0^{h} = x\\
    &X_{(i+1)h}^{(h)} = X_{ih}^{(h)} + b(ih,X_{ih}^{(h)}) h + \sigma(ih,X_{ih}^{(h)})(W_{(i+1)h}-W_{ih}).
    \end{cases}
\end{align}
%
We will use this discretised version of $X_t$ in the scheme for $Y_t=u(t,X_t)$, or similarly in the scheme for $u(t,x)$. For simplicity, we will here consider the case where $g$ is a function of $t,x,$ and $u$, but not of $\nabla u$.

By using the tower property of expectation on \autoref{eq:nonlinear-fc-for-y}, we can express $Y_{t_i}$ in terms of $Y_{t_{i+1}}$, and in terms of $X_t$ between $t_i$ and $t_{i+1}$ as
%
\begin{align} 
    Y_{t_i} = \mathbb{E}\bigg[ Y_{t_{i+1}} + \int_{t_{i}}^{t_{i+1}}g(s,X_s,Y_s)ds \Big\lvert X_{t_i} \bigg].
\end{align}
%
We recall that we are solving for $Y_t$ backwards in time, so for $t_i < t_{i+1}$, $Y_{t_{i+1}}$ is known.

From the above, we can get the (backward) Euler scheme for $Y$, namely
%
\begin{align} % not entirely sure we're conditioning on the right thing here? But seems consistent with the below...
    \begin{cases}
    &Y_T^{(h)} = f(X_T^{h})\\
    &Y_{ih}^{(h)} = \mathbb{E}\bigg[ Y_{(i+1)h}^{(h)} + h g(t_i,X_{ih}^{(h)},Y_{(i+1)h}^{(h)}) \Big\lvert X_{t_i}^{(h)} \bigg].
    \end{cases}
\end{align}
%

By conditioning on the discretised process $X^{(h)}$ starting at a specific value $x$ at time $t_i$, we get the same scheme for $u(t,x)$:
%
\begin{align}
    \begin{cases}
    &u^{(h)}(T,x) = f(x)\\
    &u^{(h)}(t_i,x) = \mathbb{E}\bigg[ u^{(h)}(t_{i+1},X^{(h),t_i,x}_{t_{i+1}}) + h g\big( t_i,x,u^{(h)}(t_{i+1},X^{(h),t_i,x}_{t_{i+1}}) \big) \bigg]. 
    \end{cases}
\end{align} 
%

We note here that the accuracy of a numerical simulation of $u(t,x)$ (or $Y_t$) depends on both the scheme we choose for $X_t$, and the scheme we choose for $Y_t$. In \cite{fang2023strong} a strong stability preserving multistep scheme, which improves the latter, is introduced. This scheme is given by
%
\begin{align} % Is this in fact right? look at X-term
    Y_{ih}^{(h)} = \sum_{j=1}^{k} \alpha_j \mathbb{E}\bigg[ Y_{(i+j)h}^{(h)} \Big\lvert X_{t_i}^{(h)} \bigg] + h \sum_{j=1}^{k} \beta_j \mathbb{E}\bigg[ g\big(t_{i+j},X_{(i+j)h}^{(h)},Y_{(i+j)h}^{(h)}\big) \Big\lvert X_{t_i}^{(h)} \bigg].
\end{align}
%
The corresponding scheme for $u(t,x)$ becomes
%
\begin{align} 
    u^{(h)}(t_i,x) = \sum_{j=1}^{k} \alpha_j \mathbb{E}\bigg[ u^{(h)}\Big(t_{i+j},X_{t_{i+j}}^{(h),t_i,x}\Big) \bigg] + h \sum_{j=1}^{k} \beta_j \mathbb{E}\bigg[ g\Big(t_{i+j},X_{t_{i+j}}^{(h),t_i,x},u^{(h)}\big(t_{i+j},X_{t_{i+j}}^{(h),t_i,x}\big)\Big) \bigg].
\end{align}
%

\subsection{Connecting the treatment planning problem and the FBSDE}

Here, we seek to connect the coupled PDEs that arise from the treatment planning problem to the framework of FBSDEs.

The PDE system arising from the treatment planning problem (with $u^{[1]}$ denoting the primal, and $u^{[2]}$ the dual) is given by
%
\begin{align}
    \begin{cases}
    \partial_t u^{[1]} - \mu \Delta u^{[1]} + b \cdot \nabla u^{[1]} &= \frac{1}{\alpha} u^{[2]} \\
    - \partial_t u^{[2]} - \mu \Delta u^{[2]} - b \cdot \nabla u^{[2]} &= u^{[1]} - d_T.
    \end{cases}
\end{align}
%
We also have the boundary condition $u^{[1]}=u^{[2]}=0$ on the boundary. 
%Here, could we say that the primal equation is solved forward in time, and the dual backward in time?
%
For simplicity, we consider only one spatial dimension
%
\begin{align}
    \begin{cases}
    \partial_t u^{[1]} - \mu \partial^2_x u^{[1]} + b \partial_x u^{[1]} &= \frac{1}{\alpha} u^{[2]} \\
    - \partial_t u^{[2]} - \mu \partial_x^2 u^{[2]} - b \partial_x u^{[2]} &= u^{[1]} - d_T.
    \end{cases}
\end{align}
%

Rewriting this to be of the form of \autoref{eq:pde-system}:
%
\begin{align}
    \begin{cases}
    \partial_t u^{[1]} - \mu \partial^2_x u^{[1]} + b \partial_x u^{[1]} - \frac{1}{\alpha} u^{[2]} &= 0 \\
    \partial_t u^{[2]} + \mu \partial_x^2 u^{[2]} + b \partial_x u^{[2]} - (u^{[1]} - d_T) &= 0.
    \end{cases}
\end{align}
%
This is equivalent to 
%
\begin{align} 
    \begin{cases}
    \partial_t u^{[1]} + \mathcal{L}^{[1]} u^{[1]} + g^{[1]} &= 0\\
    \partial_t u^{[2]} + \mathcal{L}^{[2]} u^{[2]} + g^{[2]} &= 0
    \end{cases}
\end{align}
%
for 
%
\begin{align} 
    % \begin{cases}
    &\mathcal{L}^{[1]} = - \mu \partial_x^2 + b \partial_x\\
    &\mathcal{L}^{[2]} = \mu \partial_x^2 + b \partial_x
    % \end{cases}
\end{align}
%
and 
%
\begin{align} 
    % \begin{cases}
    &g^{[1]} = -\frac{1}{\alpha} u^{[2]}\\
    &g^{[2]} = -(u^{[1]}-d_T).
    % \end{cases}
\end{align}
%
Questions/uncertainties:
\begin{itemize}
    \item $\mathcal{L}^{[1]}$ cannot be the generator of a diffusion, as this would require $\frac{1}{2}\sigma^2 = -\mu$, i.e. a diffusion coefficient $\sigma = \sqrt{-2\mu}$, which would be imaginary
    \item If we could time-reverse the forward equation (change the sign of the $\partial_t$-term) we'd maybe get the form we want (since the sign of the drift coefficient $b$ doesn't matter), but does this actually make sense to do?
    \item In the PDEs resulting from the treatment planning problem, is the primal forward in time and the dual backward in time? If so, maybe it would make sense to time-reverse the primal equation, since the FBSDE framework works with multiple equations backward in time (with terminal conditions)?
    \item If the above works out, what should the terminal condition(s)/initial condition(s) be? I.e. what are the functions $f^{[k]}(x)$?
\end{itemize}

% \subsection{Simplified toy example}

% \bibliographystyle{plain}
% \bibliography{bibliography.bib}
\nocite{*}
\printbibliography
%
\end{document}

\section{Future work}
\begin{itemize}
    \item Show convergence of $(u^{(k)},z^{(k)})$ to the true solution $(u^{(*)},z^{(*)})$
    \item Show convergence of $(u^{(k)},z^{(k)})$ to something (not neccessarily the true solution)
    \item Extend our model to multiple dimensions in space
    \item Consider other norms? FBSDE? 
    \item Implement/find/come up with better numerical methods: e.g. Strong stability Preserving (SSP) methods, Multilevel Monte Carlo (MLMC), and maybe parallelize the Monte Carlo code (each "point" (t,x,v) could be run in parallel?) GPU or FPGA
\end{itemize}

In this section, we discuss possible extensions of the material presented in this report, and outline some suggested directions for future work. First, we discuss showing convergence of the iterative procedure introduced in \autoref{sec:iterative-scheme}. Second, we discuss how the sensitivity of the convergence may be affected by the choice of methods and parameters for the numerical approximation of the PDE solutions. Then, we outline some possible ways to improve our current numerical methods. Finally, we discuss extensions of our current radiation transport model and optimisation problem.

It is not immediately clear whether the iterative scheme introduced in \autoref{sec:iterative-scheme} converges, whether if converges to the right thing, and if it does converge, in what sense it does so. One potential point of future work would be to firstly, show that the iterative scheme converges when considered in terms of the true (non-discretised) solutions to the PDEs for $u$ and $z$. This would amount to showing that $\lVert u^{(k)} - u^{(k-1)} \rVert$, goes to zero as $k$ goes to infinity, and similarly for $\lVert z^{(k)} - z^{(k-1)} \rVert$. Inspired by the results of the numerical experiment presented in \autoref{sec:numerical}, we theorize that this convergence should follow a power law, and suggest that a suitable norm to show convergence in might be the $L^2([0,T],L^2(\Omega))$-norm chosen for the optimisation problem. We would also want to show that $u^{(k)}$ and $z^{(k)}$ converge to the right thing, i.e. to the functions $u^{*}$ and $g^{*}$ which yield the minimum of \autoref{eq:to-minimise}. In this case, we'd want to show that $\lVert u^{(k)} - u^{*} \rVert$ and $\lVert \frac{1}{\alpha}z^{(k)} - g^{*} \rVert$ go to zero as $k$ goes to infinity.

After establishing convergence in the above case, we would want to consider how to establish convergence of the iterative procedure when it involves numerical approximations of $u^{(k)}$ and $z^{(k)}$ at each iteration. Relevant questions here include under what conditions the iterative procedure converges at all for the approximations of $u^{(k)}$ and $z^{(k)}$, and how close the functions it converges to are to the values at gridpoints of $u^{*}$ and $g^{*}$ defined above. We have an indication of the answer to the first of these questions from the results of the numerical experiement in \autoref{sec:numerical}. In more detail, the second question may be answered by investigating how the discretisation error (and in the case of $z^{(k)}$, Monte Carlo error) propagates through iterations. We also suggest investigating how the convergence rates of the respective numerical schemes for the primal and dual equations affect the closeness of (the approximation of) $u^{(k)}$ to $u^{*}$.

For the numerical simulations carried out in \autoref{sec:numerical}, we used a fairly simple finite difference scheme for the primal equation, and a straightforward Monte Carlo simulation based on an Euler-Maruyama discretisation for the dual equation. To improve upon the speed and accuracy of the algorithm, me may wish to consider both numerical methods with higher orders of convergence, and (on the Monte Carlo side) variance reduction techniques. For example, we could use a higher order finite difference scheme, as well as a higher order scheme (such as e.g. Milstein's scheme) to simulate the SDE. In terms of variance reduction techniques, we could apply a multilevel Monte Carlo (MLMC) technique -- see e.g. \cite{giles2008multilevel} or \cite{gobet2016monte}. As previously mentioned, for optimising the speed of our algorithm, we also recommend serialising the Monte Carlo simulations which yield the numerical approximation of $z^{(k)}$ across a grid in $t,x,v$, for each $k$. Inspiration can for example be taken from \cite{jia2012gpu}, where GPU accelerated Monte Carlo code is used for radiation dose calculations for proton beam therapy.

Additionally we may wish to specifically consider strong stability preserving numerical (SSP) methods, also known as total variation diminishing (TVD) methods, which preserve the monotonicity of a solution in space between timesteps. These methods are useful when simulating conservation laws or physical systems in general, as they yield numerical solutions that still obey the laws of physics. We refer to e.g. \cite{gottlieb2001strong}, \cite{gottlieb2001strong} and \cite{harten1983upstream} for descriptions of SSP methods in the PDE setting, and to \cite{fang2023strong} for an SSP method formulated in the setting of FBSDE and nonlinear Feynman-Kac formulae. 

Finally, the radiation transport model \autoref{eq:pde-model} that has been considered in this report only features one spatial dimension. Since radiotherapy treatment naturally takes place in three dimensional space, a neccessary extension of our model for application in the real world, is extending the radiation model to three-dimensional space. We may also wish to further consider which norm it is most suitable to formulate the minimisation problem of \autoref{eq:to-minimise}. In the case where the choice of norm yields coupled nonlinear PDEs, e.g. those of \autoref{eq:nonlinear-pde-system}, we will then need to apply the theory of FBSDE and nonlinear Feynman-Kac formulae discussed in \autoref{sec:fbsde-theory} to obtain a probabilistic representation of the solution to the dual equation. We may also wish to apply the SSP methods for FBSDE from \cite{fang2023strong}, which we briefly outline in \autoref{sec:fbsde-numerics}.